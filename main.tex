\begin{abstract}
    本文建立了描述信息学比赛的数学模型,并基于该模型研究了过往比赛的选手分数数据。本文通过统计确定了同一名选手的得分波动所服从的分布,基于此从NOIP分数推算出了选手整体水平的分布情况,并分析了信息学竞赛选拔流程的效率、回答了有关比赛名次与得分的问题。\emph{上一句待完成后更新。}本文中得到的结论对信息学竞赛赛制的优化、选手的日常训练和比赛策略制定具有参考意义。
\end{abstract}

\section{引言}

    信息学竞赛的参赛人数和竞赛水平在最近十年中快速提高。这种迅猛的发展在让竞赛趋于繁盛的同时,也使得选手和教练对竞赛现状的认知难以跟上节拍。由此引起的问题包括:

    \begin{asparaitem}
        \item {选手对于自己处于哪个水平段认识不足,从而作出错误的学业规划。}
        \item {出题人对于选手的水平认识不足,导致题目难度和部分分分配失当。}
        \item {选手不了解对手的水平和发挥情况,导致选择了错误的考场策略。}
    \end{asparaitem}

    本文将利用数学工具,基于过往比赛的选手分数数据来分析信息学竞赛的现状,以为上述问题的解决提供助力。

    正文分为五个部分:

    \begin{asparaenum}
        \item [\textbf{第二节}]{建立用于描述信息学比赛的数学模型,作为后续分析的基础。}
        \item [\textbf{第三节}]{分析同一名选手的得分波动所服从的分布。}
        \item [\textbf{第四节}]{利用NOIP初赛、复赛的得分数据推算出信息学竞赛选手整体水平的分布情况。}
        \item [\textbf{第五节}]{\emph{待完成后更新。}}
        \item [\textbf{第六节}]{\emph{待完成后更新。}}
    \end{asparaenum}