\begin{abstract}
    本文建立了描述信息学比赛的数学模型,并基于该模型研究了过往比赛的选手分数数据。本文通过统计确定了同一名选手的得分波动所服从的分布,基于此从NOIP分数推算出了选手整体水平的分布情况,并分析了信息学竞赛选拔流程的效率、回答了有关比赛名次与得分的问题。\emph{上一句待完成后更新。}本文中得到的结论对信息学竞赛赛制的优化、选手的日常训练和比赛策略制定具有参考意义。
\end{abstract}

\section{引言}

    信息学竞赛的参赛人数和竞赛水平在最近十年中快速提高。这种迅猛的发展在让竞赛趋于繁盛的同时,也使得选手和教练对竞赛现状的认知难以跟上节拍。由此引起的问题包括:

    \begin{asparaitem}
        \item {选手对于自己处于哪个水平段认识不足,从而作出错误的学业规划。}
        \item {出题人对于选手的水平认识不足,导致题目难度和部分分分配失当。}
        \item {选手不了解对手的水平和发挥情况,导致选择了错误的考场策略。}
    \end{asparaitem}

    本文将利用数学工具,基于过往比赛的选手分数数据来分析信息学竞赛的现状,以为上述问题的解决提供助力。

    正文分为五个部分:

    \begin{asparaenum}
        \item [\textbf{第二节}]{建立用于描述信息学比赛的数学模型,作为后续分析的基础。}
        \item [\textbf{第三节}]{分析同一名选手的得分波动所服从的分布。}
        \item [\textbf{第四节}]{利用NOIP初赛、复赛的得分数据推算出信息学竞赛选手整体水平的分布情况。}
        \item [\textbf{第五节}]{\emph{待完成后更新。}}
        \item [\textbf{第六节}]{\emph{待完成后更新。}}
    \end{asparaenum}

\section{建立模型}

    \subsection{赛程和赛制}

        在引入数学模型前,先对信息学竞赛的竞赛流程和比赛形式作简要介绍\footnote{赛程和赛制在近几年有小幅变化,本节中会尽量兼顾新旧两套机制}。

        信息学竞赛是一系列比赛的统称。这些比赛整体上呈现“逐级递进”的关系,即下一层比赛的优胜者晋级上一层比赛。这些比赛按照级别从低到高,大致排列为:

        \begin{asparaenum}[a.]
            \item 全国联赛(NOIP/CSP)\ ---\ 初赛 
            \item 全国联赛(NOIP/CSP)\ ---\ 复赛
            \item 省队选拔赛
            \item 清华/北大学科营(THUWC/PKUWC/THUSC/PKUSC)
            \item 亚太地区竞赛(APIO)
            \item 国家队选拔赛(CTSC/CTS)\ ---\ 非正式选手
            \item 全国冬令营(NOIWC)\ ---\ 非正式选手
            \item 全国决赛(NOI)
            \item 清华/北大集训(CTT)
            \item 全国冬令营(NOIWC)\ ---\ 正式选手
            \item 国家队选拔赛(CTSC/CTS)\ ---\ 正式选手
            \item 国际奥林匹克竞赛(IOI)
        \end{asparaenum}
        
        以下将用字母编号来代替对应的比赛。

        \begin{figure}[hb]
            \centering
            \begin{tikzcd}[column sep=large]
                {} \arrow[r, ">100000"] & a \arrow[d, "20000"]                                     &     \\
                                        & b \arrow[ld, "500"'] \arrow[d, "1000"] \arrow[rd, "500"] &     \\
                d                       & c \arrow[d, "300"] \arrow[l, "300"]                      & efg \\
                                        & h \arrow[d, "50"]                                        &     \\
                                        & ijk \arrow[d, "4"]                                       &     \\
                                        & l                                                        &                 
            \end{tikzcd}
            \caption{信息学比赛间的关系}
            \label{fig:contests}
        \end{figure}

        图\ref{fig:contests}展示了这些比赛间的关系。箭头从低级别比赛指向高级别比赛,表示该低级别比赛的优胜者可以晋级对应的高级别比赛,箭头上标记的数值表示大致晋级人数。

        赛制即比赛的进行方式和比赛规则。信息学竞赛中采用笔试、COI赛制(机试)、IOI赛制(机试)这三种不同的赛制,表\ref{tab:formats}给出了每种赛制的特点和先前提到的比赛所分别采用的赛制。

        \begin{table}[hb]\small
            \centering
            \begin{tabular}{@{}llllll@{}}
                \toprule
                    & 时长                  & 题数                 & 题目类型       & 反馈机制       & 对应比赛   \\ \midrule
                笔试  & 1\textasciitilde 2h & 数十                 & 选择题、填空题     & 无反馈        & a      \\
                COI赛制 & 3\textasciitilde 5h & 3\textasciitilde 4 & 编程题,有多档部分分 & 无反馈        & bcfghk \\
                IOI赛制 & 3\textasciitilde 5h & 3\textasciitilde 4 & 编程题,有多档部分分 & 多次提交、有反馈 & deijl  \\ \bottomrule
            \end{tabular}
            \caption{信息学比赛采用的赛制}
            \label{tab:formats}
        \end{table}

    \subsection{数学模型}

        本节将建立用于描述一场信息学比赛的数学模型。为了更清晰地界定数学模型在现实中的适用范围,需要先明确:现实中怎样的对象能被称为一场“比赛”。

        \begin{definition}[现实比赛]
            一场\textbf{现实比赛},即特定的人群在同样的规则下测试同一套题目的过程。
            一场现实比赛被\emph{参赛人群}、\emph{规则}和\emph{题目}这三个要素所确定。
        \end{definition}

        在这一定义下,每年中的 $a\sim l$ 这12个信息学比赛,自然都是现实比赛。需要注意的是,上述定义中提到的“人群”,不必包括全体参赛选手;例如若忽略NOIP初赛中所有参赛的男生,则剩下的女生依然能构成现实比赛。这一点对于后文中跨越不同比赛的分析大有帮助。
        
        接下来定义从现实比赛抽象而来的数学模型。

        \begin{definition}[理想比赛]
            \textbf{理想比赛} $A$ 由二元函数 $p_A:[0,1]\times\mathbb{R}\to\mathbb{R}_{\geq 0}$ 确定\footnote{ $\mathbb{R}_{\geq 0}$ 表示非负实数集合},其中 $p_A$ 连续且满足
            
            $$
            \int\limits_0^1\left(\int\limits_{-\infty}^\infty p_A(x,\delta)\mathrm{d}\delta\right)\mathrm{d}x=1
            $$

            此时我们把 $p_A$ 称为 $A$ 的\textbf{联合分布函数}。
        \end{definition}
