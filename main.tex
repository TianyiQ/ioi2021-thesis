\begin{abstract}
    本文建立了描述信息学比赛的数学模型,并基于该模型研究了过往比赛的选手分数数据。本文通过统计确定了同一名选手的得分波动所服从的分布,基于此从NOIP分数推算出了选手整体水平的分布情况,并分析了信息学竞赛选拔流程的效率、回答了有关比赛名次与得分的问题。\emph{上一句待完成后更新。}本文中得到的结论对信息学竞赛赛制的优化、选手的日常训练和比赛策略制定具有参考意义。
\end{abstract}

\section{引言}

    信息学竞赛的参赛人数和竞赛水平在最近十年中快速提高。这种迅猛的发展在让竞赛趋于繁盛的同时,也使得选手和教练对竞赛现状的认知难以跟上节拍。
    
    这一情况引起了一些问题,例如:

    \begin{asparaitem}
        \item {选手对于自己所处的水平段认识不足,从而作出错误的学业规划。}
        \item {出题人对于选手的水平认识不足,导致题目难度和部分分分配失当。}
        \item {选手不了解对手的水平和发挥情况,导致选择了错误的考场策略。}
    \end{asparaitem}

    本文将利用数学工具,基于过往比赛的选手分数数据来分析信息学竞赛的现状,以为上述问题的解决提供助力。

    \vspace{1.5ex}

    正文分为五个部分:

    \begin{asparaenum}
        \item [\textbf{第二节}]{建立用于描述信息学比赛的数学模型,作为后续分析的基础。}
        \item [\textbf{第三节}]{分析同一名选手的得分波动所服从的分布。}
        \item [\textbf{第四节}]{利用NOIP初赛、复赛的得分数据推算出信息学竞赛选手整体水平的分布情况。}
        \item [\textbf{第五节}]{\emph{待完成后更新。}}
        \item [\textbf{第六节}]{\emph{待完成后更新。}}
    \end{asparaenum}

\section{建立模型}

    \subsection{赛程和赛制}

        在引入模型前,先对信息学竞赛的竞赛流程和比赛形式作简要介绍\footnote{赛程和赛制在近几年有小幅变化,本小节中会尽量兼顾新旧两套机制}。

        信息学竞赛是一系列比赛的统称。这些比赛整体上呈现“逐级递进”的关系,即下一层比赛的优胜者晋级上一层比赛。这些比赛按照级别从低到高,大致排列为\footnote{后文将用下表中的字母编号来代指对应的比赛}:

        \begin{samepage}
            \begin{asparaenum}[a.]
                \item 全国联赛(NOIP/CSP)\ ---\ 初赛 
                \nopagebreak
                \item 全国联赛(NOIP/CSP)\ ---\ 复赛
                \nopagebreak
                \item 省队选拔赛
                \nopagebreak
                \item 清华/北大学科营(THUWC/PKUWC/THUSC/PKUSC)
                \nopagebreak
                \item 亚太地区竞赛(APIO)
                \nopagebreak
                \item 国家队选拔赛(CTSC/CTS)\ ---\ 非正式选手
                \nopagebreak
                \item 全国冬令营(NOIWC)\ ---\ 非正式选手
                \nopagebreak
                \item 全国决赛(NOI)
                \nopagebreak
                \item 清华/北大集训(CTT)
                \nopagebreak
                \item 全国冬令营(NOIWC)\ ---\ 正式选手
                \nopagebreak
                \item 国家队选拔赛(CTSC/CTS)\ ---\ 正式选手
                \nopagebreak
                \item 国际奥林匹克竞赛(IOI)
            \end{asparaenum}
        \end{samepage}

        \begin{figure}
            \centering
            \begin{tikzcd}[column sep=large]
                {} \arrow[r, ">100000"] & a \arrow[d, "20000"]                                     &     \\
                                        & b \arrow[ld, "500"'] \arrow[d, "1000"] \arrow[rd, "500"] &     \\
                d                       & c \arrow[d, "300"] \arrow[l, "300"]                      & efg \\
                                        & h \arrow[d, "50"]                                        &     \\
                                        & ijk \arrow[d, "4"]                                       &     \\
                                        & l                                                        &                 
            \end{tikzcd}
            \caption{信息学比赛间的关系}
            \label{fig:contests}
        \end{figure}

        图\ref{fig:contests}展示了这些比赛间的关系。箭头从低级别比赛指向高级别比赛,表示该低级别比赛的优胜者可以晋级对应的高级别比赛,箭头上标记的数值表示大致晋级人数。

        \vspace{1.5ex}

        赛制即比赛的进行方式和比赛规则。信息学竞赛中采用笔试、COI赛制(机试)、IOI赛制(机试)这三种不同的赛制,表\ref{tab:formats}给出了每种赛制的特点和先前提到的比赛所分别采用的赛制。

        \begin{table}\footnotesize
            \centering
            \begin{tabular}{@{}llllll@{}}
                \toprule
                    & 时长                  & 题数                 & 题目类型       & 反馈机制       & 对应比赛   \\ \midrule
                笔试  & 1\textasciitilde 2h & 数十                 & 选择题、填空题     & 无反馈        & a      \\
                COI赛制 & 3\textasciitilde 5h & 3\textasciitilde 4 & 编程题,有多档部分分 & 无反馈        & bcfghk \\
                IOI赛制 & 3\textasciitilde 5h & 3\textasciitilde 4 & 编程题,有多档部分分 & 多次提交、有反馈 & deijl  \\ \bottomrule
            \end{tabular}
            \caption{信息学比赛采用的赛制}
            \label{tab:formats}
        \end{table}

    \subsection{数学模型}

        本小节中将建立用于描述一场信息学比赛的数学模型。

        \subsubsection{基本模型}
        
            为了更清晰地界定模型在现实中的适用范围,需要先明确:现实中怎样的对象能被称为一场“比赛”。

            \begin{definition}[现实比赛]
                一个\textbf{现实比赛},即特定的人群在同样的规则下测试同一套题目的过程。
                一个现实比赛被\emph{参赛人群}、\emph{规则}和\emph{题目}这三个要素所确定。
                \label{def:realContest}
            \end{definition}

            在这一定义下,每年中的 $a\sim l$ 这12个信息学比赛,自然都是现实比赛。
            
            \vspace{1.5ex}

            关于“参赛人群”这一概念需要注意两点:
            
            \begin{asparaitem}
                \item 参赛人群只是一个宽泛的范围,而不是具体的选手集合。例如我们可以规定参赛人群为“所有学习信息学的同学”,但这一规定并不关注张三、李四、王五是否是这个人群的成员。这样的规定不会给后续的分析带来不利影响,因为我们只关心关于比赛和人群的统计信息,而不关心每名选手的特点。
                \item 参赛人群不必囊括实际参赛的整个选手群体;例如在NOIP初赛中,“所有报名了初赛的女生”这一参赛群体依然能构成现实比赛。这一点对于后文中跨越不同比赛的分析大有帮助。
            \end{asparaitem}
            
            \vspace{1.5ex}

            接下来定义从现实比赛抽象而来的数学模型。

            \begin{definition}[理想比赛]
                \textbf{理想比赛} $A$ 由二元函数 $H_A:\left[0,1\right]\times\mathbb{R}\to\mathbb{R}_{\geq 0}$ 确定,其中 $H_A$ 连续且满足
                
                $$
                \int\limits_0^1\int\limits_{-\infty}^\infty H_A(x,\delta)\mathrm{d}\delta\mathrm{d}x=1
                $$

                此时我们把 $H_A$ 称为 $A$ 的\textbf{综合分布函数}。
                \label{def:idealContest}
            \end{definition}

            接下来将定义:一个理想比赛何时被认为“描述”了一个现实比赛。这也将同时表明综合分布函数的实际含义。
            
            \vspace{1.5ex}

            首先约定一下记号:

            \begin{asparaitem}
                \item $\mathrm{Pr}\left[A\right]$ 表示事件 $A$ 发生的概率。
                \item $\mathrm{E}\left[X\right]$ 表示随机变量 $X$ 的期望值。
            \end{asparaitem}

            \begin{definition}
                从现实比赛 $B$ 可按如下方式确定一个理想比赛 $A$ :

                \begin{asparaenum}[\bfseries{步骤} 1.]
                    \item 记 $B$ 的参赛选手集合为有限集 $S_B$ ,并在 $B$ 的参赛人群(包括人群内部的具体构成)不变的情况下,假想参赛人数 $\left|S_B\right|$ 趋于无穷。我们之所以能够任意钦定 $\left|S_B\right|$ ,是因为——如先前所述——$B$的定义并未指明具体的选手集合。
                    \item 每一名参赛选手 $p$ 在比赛 $B$ 中的实际得分 $\textit{score}_p$ 是一个随机变量,它被各种偶然因素(如临场发挥)所支配,但是它的分布可以由选手 $p$ 和现实比赛 $B$ 的三个要素完全确定。假想对每一名选手 $p$ 计算其期望得分 $\textit{exscore}_p=\mathrm{E}\left[\textit{score}_p\right]$ ,并取所有选手期望得分的最大值,记作 $M_B$ 。由于参赛人数趋于无穷,每一个个人的特征可以忽略,故 $M_B=\max\limits_{p\in S_B} \textit{exscore}_p$ 仅由 $B$ 确定。
                    \item 从 $S_B$ 中等概率随机选取一名选手 $p$ ,并:
                    \begin{itemize}[leftmargin=4em]
                        \item [$\bullet$] 定义 $[0,1]$ 上的随机变量 $X_B=\dfrac{\textit{exscore}_p}{M_B}$ 。\footnote{“将每名选手的分数除以最高分数”这一操作,类似于信息学比赛中计算标准分的方式}易见随机变量 $X_B$ 的实际取值与考场上的偶然因素无关,而是由选取 $p$ 的方式确定。
                        \item [$\bullet$] 定义 $\mathbb{R}$ 上的随机变量 $\Delta_B=\textit{score}_p-\textit{exscore}_p$ 。易见随机变量 $\Delta_B$ 的实际取值由选取 $p$ 的方式和考场上的偶然因素(如选手临场发挥)共同确定。
                        \item [$\circ$] 请注意,$X_B$ 和 $Delta_B$ 的定义中所用的 $p$ 是\emph{同一名}随机选择的选手,而不是独立的两次选择。
                    \end{itemize}
                    \item 取 $A$ 的综合分布函数 $H_A$ 为 $X_B$ 与 $\Delta_B$ 的联合概率密度函数,从而确定 $A$ 。换句话说,对所有 $X_0\in[0,1],\Delta_0\in\mathbb{R}$ ,需要满足\footnote{也可以直观地理解为:$H_A(X_0,\Delta_0)=\mathrm{Pr}\left[\left(X_B\approx X_0\right)\land\left(\Delta_B\approx\Delta_0\right)\right]$}
                    
                    $$
                    \int\limits_0^{X_0}\int\limits_{-\infty}^{\Delta_0} H_A(x,\delta)\mathrm{d}\delta\mathrm{d}x=\mathrm{Pr}\left[\left(X_B\leq X_0\right)\land\left(\Delta_B\leq\Delta_0\right)\right]
                    $$

                \end{asparaenum}

                对于按上述方式得到的 $A$ ,我们称 $A$ 与 $B$ \textbf{互相对应}。
                \label{def:realIdealCorrespondence}
            \end{definition}

            冗长的定义可以用一句话来作直观的总结:$H_A\left(x,\delta\right)$ 表示真实水平(即期望得分)约为 $x$ ($x\in\left[0,1\right]$为按最高分折算后的标准分)、实际表现约为 $x+\delta$ (同样表示标准分)的选手的\emph{期望}人数占总人数的比例;之所以实际表现会偏离真实水平——以及这里之所以说“期望人数”——是因为考场上的各种偶然因素为比赛结果带来了随机性。

            可以看到,理想比赛这一模型只考虑了哪些结果\emph{可能}出现,而未考虑哪种结果\emph{实际}出现。而在现实中,能够获知的却只有实际出现的结果——和它恰恰相反。下面定义的概念将处理这一问题。

            \begin{definition}[分数分布函数]
                对理想比赛 $A$ ,定义其\textbf{分数分布函数} $C_A:\mathbb{R}\to\mathbb{R}_{\geq 0}$ 满足

                $$
                C_A(s)=\int\limits_{0}^1 H_A(x,s-x) \mathrm{d}x
                $$

                \label{def:scoreDistribution}
            \end{definition}

            \begin{proposition}[分数分布函数的实际含义]
                对现实比赛 $B$ 和与之对应的理想比赛 $A$ ,假想比赛 $B$ 的参赛人数 $|S_B|$ 趋于无穷,等概率随机选取选手 $p\in S_B$ ,则 $\mathrm{Pr}\left[\textit{score}_p\leq r\right]=\int_{-\infty}^r C_A(s)\mathrm{d}s,\forall r\in\mathbb{R}$ 。\footnote{和先前类似,这里也可以直观理解为 $C_A(r)=\mathrm{Pr}\left[\textit{score}_p\approx r\right]$ }
                \label{prop:scoreDistributionMeaning}
            \end{proposition}

            \begin{proof}
                \begin{align}
                    \mathrm{Pr}\left[\textit{score}_p\leq r\right]
                    &=\mathrm{Pr}\left[X_B+\Delta_B\leq r\right] \notag \\
                    &=\iint\limits_{\left\{(x,\delta):x\in\left[0,1\right],\delta\in\mathbb{R},x+\delta\leq r\right\}}H_A(x,\delta)\mathrm{d}(x,\delta) \notag \\
                    &=\iint\limits_{\left\{(x,s):x\in\left[0,1\right],s\in\left(-\infty,r\right]\right\}}H_A(x,s-x)\mathrm{d}(x,s) \notag \\
                    &=\int\limits_{-\infty}^r\left(\int\limits_0^1 H_A(x,s-x)\mathrm{d}x\right)\mathrm{d}s \notag \\
                    &=\int\limits_{-\infty}^r C_A(s)\mathrm{d}s \notag 
                \end{align}
            \end{proof}

            在上面四个定义中,涉及到现实情况的部分难免有模糊之处;实际应用中对这几条定义的执行,也不可避免地需要作近似处理。但即便如此,作出这些规定依然能极大地帮助我们厘清思路并发现隐含的前提。

        \subsubsection{特殊情况下的模型}

            在一场现实比赛中,对每个选手 $p$ ,$p$ 的实际得分相比真实水平的“得分偏移量” $\textit{score}_p-\textit{exscore}_p$ 都是一个随机变量。如果所有选手的“得分偏移量”独立同分布,对我们的模型意味着什么?

        \subsubsection{几个关键的假设}

